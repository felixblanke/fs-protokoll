\documentclass[
    gremium=FSR, % Gremium: FSR, FSV oder weglassen
    english,      % Sprache: ngerman oder english
    wideoverview, % weite Übersicht anschalten
    %raggedtitle, % rechtsbündigen Titel anschalten
    %draft,       % als Entwurf kennzeichnen
]{fs-protokoll}


\date{XX.XX.2021}
\leiter{Erika Musterfrau} %Sitzungsleitung
\protokollant{Max Mustermann}
\beginn{ Uhr}
\Ende{ Uhr}
\mitglieder{Kim \enquote{Toller Spitzname} Nguyen}
\fehlende{-} %Unentschuldigt fehlende Mitglieder des Organs

\entfehlende{} %Entschludigt fehlende Mitglieder des Organs

\anwesende{} %Anwesende, die nicht zum Organ gehören

%
% EINSTELLUNGEN:
%

% Für eine konstituierende FSV-Sitzung:
%\konstitrue

% Zeige ordentliche/außerordentliche Sitzung an
%\useordentlichetrue
% Setze in dem Fall die Sitzungsnummer:
%\sitzungsnr{1}

% Falls es eine außerordentliche Sitzung ist: 
%\ordentlichesitzungfalse

% Falls ein anderes Gremium außer FSR/FSV protokolliert wird:
%\gremiumprefix{des} % Stellt den Artikel für "Sitzung [Präfix] Gremiums" ein
%\gremiums{Wahlausschuss} % Name des protokollierten Gremiums (im Genitiv)

%weitere Einstellungen:
% \internetfalse      % Schaltet Kommentar zu Protokollen im Internet ab
% \fusszeilefalse     % Nur noch Seitenzahlen in der Fußzeile
% \titelfalse         % Macht derzeit gar nichts
% \leitungfalse       % Schaltet ab, dass es Sitzungsleitung/Protokollierende gibt
% \unterschriftfalse  % Entfernt die Unterschriftenzeile aus dem Protokoll
% \terminlistefalse   % Entfernt die Terminliste

\begin{document}

\TOP{Eröffnung und Begrüßung (Anfangsuhrzeit -- Enduhrzeit Uhr)}

\TOP{Festlegung der Tagesordnung (Anfangsuhrzeit -- Enduhrzeit Uhr)}

\TOP{Protokolle (Anfangsuhrzeit -- Enduhrzeit Uhr)}

\TOP{Berichte (Anfangsuhrzeit -- Enduhrzeit Uhr)}
\begin{description}
\item[FSR] Hier ein Bericht aus dem FSR
\item[weiterer Bericht] Hier ein toller Bericht
\item[noch ein Bericht] Ganz spannende Dinge werden berichtet.
\end{description}

\TOP{Gremien (Anfangsuhrzeit -- Enduhrzeit Uhr)}

\TOP{%Titel des Tops (Anfangsuhrzeit-Enduhrzeit Uhr)
}

%Inhalt des Tops

\enquote{Anführungsstriche} %Das hier steht in Anführungszeichen

\sitzungsbreak{endlich protokolliert wer vernünftiges} % Hervorheben von Sitzungspause, Redeleitungswechel, Protokollierendenwechsel, Wiederaufnahme eines Tops

\begin{nominationtable}
    wird nominiert        &   nominiert   & Ja\\
    das können           &   auch mehrere sein   & will aber nicht\\
\end{nominationtable}
        

% \begin{electiontable}{2 %hier steht die Anzahl der Wahlgänge
% }
%             Kandidat 1  & Stimmen im 1.Wahlgang & Stimmen 2.Wahlgang\\
%             easy win & 6 & 8\\
%             War auch dabei           & 4 & 0\\
%             Enthaltung      & 0 & 2\\
%         \end{electiontable}

\begin{antragstable}{1} %Anzahl weitere Spalten
    Antrag & Stimmen\\\midrule
    Ja          & Anzahl Stimmen\\
    Nein        & Anzahl Stimmen\\
    Enthaltung  & Anzahl Stimmen\\
\end{antragstable}

%Meinungsbild zu ...
\begin{center}
\begin{tabular}{c|c|c} 
    \textbf{Option 1} & \textbf{Option 2} & \textbf{Enthaltung}\\ \hline 
     Stimmen 1 & Stimmen 2 & Stimmen E
\end{tabular}
\end{center}


\TOP{Sonstiges}
\end{document}
