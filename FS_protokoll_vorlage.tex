\documentclass[
    gremium=FSR, % Gremium: FSR, FSV oder weglassen
    ngerman,      % Sprache: ngerman oder english
    wideoverview, % weite Übersicht anschalten
    %raggedtitle, % rechtsbündigen Titel anschalten
    %draft,       % als Entwurf kennzeichnen
    %logofile=FILENAME, % anderes Logo laden
]{fs-protokoll}


\date{XX.XX.2021}
\leiter{Erika Musterfrau} %Sitzungsleitung
\protokollant{Max Mustermann}
\beginn{ Uhr}
\Ende{ Uhr}
\anwesende{} %Anwesende

% Wird für eine FSR-Sitzung nicht gebraucht:

% \mitglieder{Kim \enquote{Toller Spitzname} Nguyen} % Mitglieder des Organs
% \fehlende{} %Unentschuldigt fehlende Mitglieder des Organs
% \entfehlende{} %Entschludigt fehlende Mitglieder des Organs


%
% EINSTELLUNGEN:
%

% Falls es ein unöffentliches Protkoll werden soll:
%\oeffentlichesprotokollfalse

% Zeige ordentliche/außerordentliche Sitzung an
%\useordentlichetrue
% Setze in dem Fall die Sitzungsnummer:
%\sitzungsnr{1}

% Falls es eine außerordentliche Sitzung ist: 
%\ordentlichesitzungfalse

% Für konstituierende Sitzungen:
%\konstitrue

% Für die konstituierende FSV-Sitzung:
%\oldfsr{} Namen des alten FSR-Vorstands (für die Unterschriften)
%\oldfsr{} Namen des neuen FSR-Vorstands (für die Unterschriften)
% Falls die Unterschrift des alten/neuen FSR-Vorsitzes aus dem konsti-Protokoll entfernt werden soll
%\fsrvorstandunterschriftfalse


% Falls ein anderes Gremium außer FSR/FSV protokolliert wird:
%\gremiumprefix{des} % Stellt den Artikel für "Sitzung [Präfix] Gremiums" ein
%\gremiums{Wahlausschuss} % Name des protokollierten Gremiums (im Genitiv)

%
% WEITERE EINSTELLUNGEN
%

% \internetfalse      % Schaltet Kommentar zu Protokollen im Internet ab
% \fusszeilefalse     % Nur noch Seitenzahlen in der Fußzeile
% \titelfalse         % Macht derzeit gar nichts
% \leitungfalse       % Schaltet ab, dass es Sitzungsleitung/Protokollierende gibt
% \unterschriftfalse  % Entfernt die Unterschriftenzeile aus dem Protokoll
% \mitgliederlistefalse % Schaltet die Aufschlüsselung in Organmitglieder aus (bei FSR standardmäßig aus)

%
% TERMINLISTE
%

% \terminlistefalse   % Entfernt die Terminliste

% \renewcommand{\termtable}{%
%     \begin{tabular}{rl}
%         07.01.2021            & Spieleabend \\
%         10.01.2021            & Matheball \\
%         17. -- 18.01.2021     & Fachschaftsfahrt \\
%         21.01.2021            & Spieleabend \\
%         24.--26.01.2021       & KoMa \\
%     \end{tabular}
% }


\begin{document}

\TOP{Eröffnung und Begrüßung (Anfangsuhrzeit -- Enduhrzeit Uhr)}

\TOP{Festlegung der Tagesordnung (Anfangsuhrzeit -- Enduhrzeit Uhr)}

\TOP{Protokolle (Anfangsuhrzeit -- Enduhrzeit Uhr)}

\TOP{Berichte (Anfangsuhrzeit -- Enduhrzeit Uhr)}
\begin{description}
\item[FSR] Hier ein Bericht aus dem FSR
\item[weiterer Bericht] Hier ein toller Bericht
\item[noch ein Bericht] Ganz spannende Dinge werden berichtet.
\end{description}

\TOP{Gremien (Anfangsuhrzeit -- Enduhrzeit Uhr)}

\TOP{%Titel des Tops (Anfangsuhrzeit-Enduhrzeit Uhr)
}

%Inhalt des Tops

\enquote{Anführungsstriche} %Das hier steht in Anführungszeichen

\sitzungsbreak{endlich protokolliert wer vernünftiges} % Hervorheben von Sitzungspause, Redeleitungswechel, Protokollierendenwechsel, Wiederaufnahme eines Tops

\begin{nominationtable}
    wird nominiert        &   nominiert   & Ja\\
    das können           &   auch mehrere sein   & will aber nicht\\
\end{nominationtable}
        

% \begin{electiontable}{2 %hier steht die Anzahl der Wahlgänge
% }
%             Kandidat 1  & Stimmen im 1.Wahlgang & Stimmen 2.Wahlgang\\
%             easy win & 6 & 8\\
%             War auch dabei           & 4 & 0\\
%             Enthaltung      & 0 & 2\\
%         \end{electiontable}

\begin{antragstable}{1} %Anzahl weitere Spalten
    Antrag & Stimmen\\\midrule
    Ja          & Anzahl Stimmen\\
    Nein        & Anzahl Stimmen\\
    Enthaltung  & Anzahl Stimmen\\
\end{antragstable}

%Meinungsbild zu ...
\begin{center}
\begin{tabular}{c|c|c} 
    \textbf{Option 1} & \textbf{Option 2} & \textbf{Enthaltung}\\ \hline 
     Stimmen 1 & Stimmen 2 & Stimmen E
\end{tabular}
\end{center}

\begin{nichtoeff}
	Dieser Teil wird nicht oeffentlich mitprotokolliert. 
\end{nichtoeff}


\TOP{Sonstiges}
\end{document}
